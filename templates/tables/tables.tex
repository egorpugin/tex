\documentclass{article}

% пакеты для языка, формул
\usepackage{amsmath, amsthm, amssymb} % formulas
\usepackage{unicode-math}
\usepackage{tinos}
%\usepackage{fontspec}
%\setmainfont{Times New Roman}
\usepackage{polyglossia} % for russian hyphenation
\setmainlanguage{russian}
\setotherlanguage{english}
\newfontfamily{\cyrillicfonttt}{Liberation Mono}

% пакеты для таблиц

% ссылки
\usepackage{hyperref}

% настройки документа
\title{Latex - примеры таблиц}
\author{Пугин Е.В.}

\begin{document}

\maketitle
\newpage

\tableofcontents
\newpage

\url{https://en.wikibooks.org/wiki/LaTeX/Tables}

\bigskip

\section{Простые табличные данные (не таблица!) без переносов между листами}



\bigskip

\begin{tabular}{ l  c  r }
Число & 123 & 123 \\
  1 & 2 & 3 \\
\end{tabular}
\bigskip

\begin{tabular}{ l  c  r }
 \hline		
Число & 123 & 123 \\
 \hline		
  1 & 2 & 3 \\
 \hline		
\end{tabular}
\bigskip

\begin{tabular}{ | l | c | r |}
 \hline		
Число & 123 & 123 \\ \hline		
  1 & 2 & 3 \\ \hline		
  4 & 5 & 6 \\ \hline		
  7 & 8 & 9 \\ \hline
  1 & 2 & 3 \\ \hline
  4 & 5 & 6 \\ \hline
\end{tabular}
\bigskip

% форматирование по левому краю (l) | центру (c) | правому краю (r)
\begin{tabular}{ l | c | r }
 \hline		
Число & 123 & 123 \\
 \hline		
  1 & 2 & 3 \\
  4 & 5 & 6 \\
  7 & 8 & 9 \\
  1 & 2 & 3 \\
  4 & 5 & 6 \\
 \hline		
\end{tabular}
\bigskip




\newpage

\section{Простая таблица (в окружении table) без переносов между листами}

Использование [h] нежелательно, но допустимо.

\begin{table}[h]
\caption{123}
\begin{tabular}{ l  c  r }
Число & 123 & 123 \\
  1 & 2 & 3 \\
\end{tabular}
\end{table}

\begin{table}[h]
\caption{123}
\begin{tabular}{ l  c  r }
 \hline		
Число & 123 & 123 \\
 \hline		
  1 & 2 & 3 \\
 \hline		
\end{tabular}
\end{table}

\begin{table}[h]
\begin{tabular}{ | l | c | r |}
 \hline		
Число & 123 & 123 \\ \hline		
  1 & 2 & 3 \\ \hline		
  4 & 5 & 6 \\ \hline		
  7 & 8 & 9 \\ \hline
  1 & 2 & 3 \\ \hline
  4 & 5 & 6 \\ \hline
\end{tabular}
\end{table}

\begin{table}[h]
\caption{по центру}
\centering
\begin{tabular}{ l | c | r }
 \hline		
Число & 123 & 123 \\
 \hline		
  1 & 2 & 3 \\
  4 & 5 & 6 \\
  7 & 8 & 9 \\
  1 & 2 & 3 \\
  4 & 5 & 6 \\
 \hline		
\end{tabular}
\end{table}

\end{document}
